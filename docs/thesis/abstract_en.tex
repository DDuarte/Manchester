\chapter*{Abstract}

% TODO Rewrite, same as initial summary

Customers interact with e-commerce websites in multiple ways and the
companies operating them rely on optimizing success metrics such as
CTR (Click through Rate), CPC (Cost per Conversion), Basket and Lifetime
Value and User Engagement for profit. Changing what, how and when content
such as product recommendations and ads are displayed can influence
customers' actions.

Multiple algorithms and techniques in data mining and machine learning
have been applied in this context. Summarizing and analyzing user
behaviour can be expensive and tricky since it's hard to extrapolate
patterns that never occurred before and the causality aspects of the
system are not usually taken into consideration. Commonly used online
techniques such as A/B testing and multi-armed bandit optimization have
the down side of having a high operational cost (including time e.g if
a data scientist is evaluating the impact of a new recommendation engine
after one month, she would need to wait an actual month to have results).
However, there has been studies about characterizing user behaviour and
interactions in e-commerce websites that could be used to improve this
process.

The goal of this dissertation is to create a framework capable of running
a multi-agent simulation, by regarding users in an e-commerce website and
react to stimuli that influence their actions. Furthermore, some statistical
constructs such as Baysian networks, Markov chains or probability
distributions can be used to guide how these agents interact with the
system. By taking input from web mining (\emph{Web structure mining} (WSM),
\emph{Web usage mining} (WUM) and \emph{Web content mining} (WCM)), which
includes both static and dynamic content of websites as well as user
personas, the simulation should collect success metrics so that the
experimentation being run can be evaluated. For example, this framework
could be used to try different approaches to product recommendation and
estimate the impact of it.


% \chapter*{Resumo}
