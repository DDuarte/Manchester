\chapter*{Abstract}

% TODO Rewrite, same as initial summary

Customers interact with e-commerce websites in multiple ways and the
companies operating them rely on optimizing success metrics for profit. 
Changing what, how and when content such as product recommendations and ads are 
displayed can influence customers' actions.

Multiple algorithms and techniques in data mining and machine learning
have been applied in this context. Summarizing and analyzing user
behaviour can be expensive and tricky since it's hard to extrapolate
patterns that never occurred before and the causality aspects of the
system are not usually taken into consideration. Commonly used online
techniques have the down side of having a high operational cost. However, there 
has been studies about characterizing user behaviour and interactions in 
e-commerce websites that could be used to improve this process.

The goal of this dissertation is to create a framework capable of running
a multi-agent simulation, by regarding users in an e-commerce website that
react to stimuli that influence their actions. Furthermore, some probabilistic 
models can be used to guide how these agents interact with the system. By 
taking input from web mining, which includes both static and dynamic content of 
websites as well as user personas, the simulation should collect success 
metrics so that the experimentation being run can be evaluated.

% \chapter*{Resumo}

% TODO: translate to portuguese
