\chapter*{Abstract}

% TODO Rewrite, same as initial summary

Customers interact with e-commerce websites in multiple ways and the
companies operating them rely on optimizing success metrics for profit. 
Changing what, how and when content such as product recommendations and ads are 
displayed can influence customers' actions.

Multiple algorithms and techniques in data mining and machine learning
have been applied in this context. Summarizing and analysing user
behaviour can be expensive and tricky since it's hard to extrapolate
patterns that never occurred before and the causality aspects of the
system are not usually taken into consideration. Commonly used online
techniques have the downside of having a high operational cost. However, there 
has been studies about characterizing user behaviour and interactions in 
e-commerce websites that could be used to improve this process.

The goal of this dissertation is to create a framework capable of running
a multi-agent simulation, by regarding users in an e-commerce website that
react to stimuli that influence their actions. By taking input from web mining, 
which includes both static and dynamic content of websites as well as user 
personas, the simulation should collect success metrics so that the 
experimentation being run can be evaluated.

\chapter*{Resumo}

Consumidores interagem com websites de comércio eletrónico de várias formas e 
as empresas que os operam dependem da otimização de métricas de sucesso tais 
como \textit{CTR} (\textit{Click through Rate}), \textit{CPC} (\textit{Cost per 
Conversion}), \textit{Basket} e \textit{Lifetime Value} e \textit{User 
Engagement} para lucro. Alterar como, onde e quando o conteúdo de páginas web 
como por exemplo recomendação de produtos e publicidade é mostrado pode 
influenciar as ações dos consumidores.

Vários algoritmos e técnicas em \textit{data mining} e \textit{machine 
learning} têm sido aplicados neste contexto. Sumarizar e analisar comportamento 
de utilizadores pode ser custoso e complicado porque é difícil extrapolar 
padrões que nunca ocorreram antes e os aspetos causais do sistema geralmente 
não são tidos em consideração. Técnicas \textit{online} geralmente usadas têm o 
problema de ter um custo operacional elevado. Porém, existem estudos sobre 
caracterizar comportamento e interações de utilizadores em sites de comércio 
eletrónico que podem ser usados para melhorar este processo.

O objetivo desta dissertação é criar uma \textit{framework} capaz de correr uma 
simulação multi-agente, tendo em conta os utilizadores de um site de comércio 
eletrónico que reagem a estímulos que influenciam as suas ações. Extraindo 
dados de web mining, que inclui tanto conteúdo estático como dinâmico de 
websites assim como de perfis de utilizadores, a simulação deve reportar 
métricas de sucesso para que a experiência possa ser avaliada.
