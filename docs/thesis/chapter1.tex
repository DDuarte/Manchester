\chapter{Introduction} \label{chap:intro}

\section*{}

In this chapter we intend to introduce the report, starting by describing its 
context, motivations and objectives that will drive the dissertation. It ends 
with a description of the report structure.

%% O primeiro capítulo da dissertação deve servir para apresentar o
%% enquadramento e a moti\-va\-ção do trabalho e para identificar e
%% definir os problemas que a dissertação aborda.
%% Deve resumir as metodologias utilizadas no trabalho e termina
%% apresentando um breve resumo de cada um dos capítulos
%% posteriores.

\section{Context} \label{sec:context}
 
%% Esta secção descreve a área em que o trabalho se insere, podendo
%% referir um eventual projeto de que faz parte e apresentar uma breve
%% descrição da empresa onde o trabalho decorreu.
%% 

Customers interact with e-commerce websites in multiple ways and the companies
operating them rely on optimizing success metrics such as CTR (Click
through Rate), CPC (Cost per Conversion), Basket and Lifetime Value and User
Engagement for profit. Changing what, how and when content such as product
recommendations and ads are displayed can influence customers' actions.

Multiple algorithms and techniques in data mining and machine learning
have been applied in this context.

\section{Motivation and Goals} \label{sec:goals}

Modelling user behaviour on the web is not a new problem. It has been applied 
with different objectives, from improving the performance of cache 
servers, to the improvement of search engine, influencing purchase patters or 
recommending related pages or products~\cite{Deshpande2001, 
JSrivastavaRCooley2000}. However, all these approaches were done with a machine 
learning mindset -- \textit{predicting} which page the user or customer will 
browse next. This requires extensive use of existing and historical training 
datasets which might not expose all the causality aspects of the system. What 
if the data (or the time needed to gather it) is simply not available?

Let's imagine that we developed a new recommendation engine algorithm. One of 
the most common ways to evaluate it is by testing the engine with \textit{A/B 
testing\footnote{formally, two-sample hypothesis testing}}, which is a 
randomized experiment where a group of users are presented with one version of 
the engine (control) and the other group is shown the improved version of the 
engine. By analysing how the two groups behave differently, it's possible to 
assess the quality of the two versions, comparatively. However, this approach 
may not be feasible in all situations. For the experiment to be statistically 
significant, the number of users shown the two versions of the product must be 
enough. The experiment also takes time to run and the metrics used to compare 
both versions might not be easy to choose. \cite{Quora2015}.

The goal of this dissertation is to create a framework capable of running a
multi-agent simulation (chapter \ref{chap:simulation}), by regarding users in 
an e-commerce website and react to stimuli that influence their actions 
(chapter \ref{chap:ecommerce}). Furthermore, some statistical constructs such 
as Baysian networks, Markov chains or probability distributions (chapter 
\ref{chap:models}) can be used to guide how these agents interact with the 
system. By taking input from web mining (Web structure mining (WSM), Web usage 
mining (WUM) and Web content mining (WCM)), which includes both static and 
dynamic content of websites as well as user personas, the simulation should 
collect success metrics so that the experimentation being run can be evaluated.

\section{Report Structure} \label{sec:struct}

Besides this introduction, this report has 5 more chapters.

In chapters \ref{chap:ecommerce}, \ref{chap:simulation} and \ref{chap:models}, 
we describe the literature review and state of the art in regard to e-commerce, 
simulation systems and probabilistic models, respectively. The chapter 
\ref{chap:ecommerce} focuses on e-commerce background, what metrics can be used 
on e-commerce websites and the customer life cycle, an important part in the 
simulation. The chapter \ref{chap:simulation} describes three main topics 
regarding simulating systems: agent based, discrete event simulation and some 
hybrid approaches. Finally, the chapter \ref{chap:models} deals with describing 
some probabilistic models, with emphasis on graphical models and Bayesian 
statistics.

Chapter \ref{chap:method} is concerned with describing the methodology and 
proposes a work plan for the dissertation.

The final chapter, chapter \ref{chap:conclusion}, concludes the work realized 
and describes a SWOT\footnote{Strengths, weaknesses, opportunities and threats} 
analysis applied to the project.
