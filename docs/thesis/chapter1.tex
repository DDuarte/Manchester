\chapter{Introduction} \label{chap:intro}

\section*{}

%% O primeiro capítulo da dissertação deve servir para apresentar o
%% enquadramento e a moti\-va\-ção do trabalho e para identificar e
%% definir os problemas que a dissertação aborda.
%% Deve resumir as metodologias utilizadas no trabalho e termina
%% apresentando um breve resumo de cada um dos capítulos
%% posteriores.
%% 
%% Este documento ilustra o formato a usar em dissertações na \Feup, não
%% servindo de exemplo sobre os conteúdos a usar.
%% São dados exemplos de margens, cabeçalhos, títulos, paginação, estilos
%% de índices, etc. 
%% São ainda dados exemplos de formatação de citações, figuras e tabelas,
%% equações, referências cruzadas, lista de referências e índices.
%% 
%% Uma recolha de normas existentes sobre este assunto pode ser
%% encontrada em~\cite{kn:Mat93}. 
%% 
%% \begin{quote}
%%   ``Like the Abstract, the Introduction should be written to engage the
%%   interest of the reader. It should also give the reader an idea of
%%   how the dissertation is structured, and in doing so, define the
%%   thread of the contents.''~\cite[chap.\ Introduction]{kn:Tha01} 
%% \end{quote}
%% 
%% Neste primeiro capítulo ilustra-se a utilização de citações e de
%% referências biblio\-grá\-fi\-cas.
%% Para além de dar um exemplo de utilização de uma citação, a citação
%% anterior, introduz uma referência que pode ser consultada, entre
%% muitas outras referências bibliográficas
%% interessantes~\cite{kn:Tha01,kn:PP05}. 

\section{Context} \label{sec:context}
 
%% Esta secção descreve a área em que o trabalho se insere, podendo
%% referir um eventual projeto de que faz parte e apresentar uma breve
%% descrição da empresa onde o trabalho decorreu.
%% 

\section{Motivation and Goals} \label{sec:goals}

%% Apresenta a motivação e enumera os objetivos do trabalho terminando
%% com um resumo das metodologias para a prossecução dos objetivos.

\section{Report Structure} \label{sec:struct}

%% Para além da introdução, esta dissertação contém mais x capítulos.
%% No capítulo~\ref{chap:sota}, é descrito o estado da arte e são
%% apresentados trabalhos relacionados. 
%% %\todoline{Complete the document structure.}
%% No capítulo~\ref{chap:chap3}, ipsum dolor sit amet, consectetuer
%% adipiscing elit.
%% No capítulo~\ref{chap:chap4} praesent sit amet sem. 
%% No capítulo~\ref{chap:concl}  posuere, ante non tristique
%% consectetuer, dui elit scelerisque augue, eu vehicula nibh nisi ac
%% est. 
%% 