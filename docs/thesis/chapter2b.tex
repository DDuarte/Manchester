\chapter{Literature Review: System Simulation} \label{chap:simulation}

\section*{}

This chapter intends to introduce some approaches to computational simulation 
systems and engines, namely agent based and discrete event simulation. To 
finish the chapter, we show some novel approaches to simulation.

\section{Introduction}

Simulations are used to reproduce the behaviour of a system. They have been 
applied to different areas like physics, weather, biology, economics and many 
others. There are many types of simulations: stochastic or deterministic, 
steady-state or dynamic, continuous or discrete and local or distributed 
\cite{WKSimulation}. These categories are not exhaustive nor exclusive.

In this literature review, we are particularly interested in studying 
simulations which can model stochastic processes and not dynamic (dynamic 
systems are usually described by differential equations and are continuous by 
definition).

\section{Agent Based Simulation (ABS)}

In agent based simulation (ABS), sometimes described as agent based computing 
\cite{wooldridge1998agent, jennings1999agent}, the individual entities in the 
model are represented discretely and maintain a set of behaviours, beliefs or 
rules that determine how their state is updated. \cite{Niazi2011} lists three 
different approaches to agent based modelling and simulation:

\begin{itemize}
    \item \emph{Agent-oriented programming} which puts emphasis on developing 
    complex individual agents rather than a large set of agents;
    \item \emph{Multi-agent oriented programming} focus on adding \emph{some} 
    intelligence to agents and observe their interactions;
    \item \emph{Agent-based or massively multi-agent modelling} where the main 
    idea is to build simple models for the agents which interact with a large 
    population of other agents to observe the global behaviour.
\end{itemize}

\cite{Siebers2010} describes ABS as ``well suited to modelling systems with 
heterogeneous, autonomous and pro-active actors, such as human-centred 
systems.'', which make them a good candidate to be used in the development of 
this dissertation. However, existing literature is quite confusing and broad, 
using different terms to refer to the same concepts, without clear 
distinctions between different agent based approaches and without consensus 
\cite{Niazi2011, Brailsford2014}.

Many platforms and frameworks were developed to support agent-based modelling 
and simulation. Some notable examples include Repast~\cite{collier2003repast}, 
NetLogo~\cite{wilensky1999netlogo}, StarLogo~\cite{resnick1996starlogo} or 
MASON~\cite{panait2005cooperative}. An updated list is maintained at OpenABM 
\cite{OpenABM2016}.

Agents have been applied to e-commerce context mostly in two distinct areas: 
recommendation systems \cite{xiao2007commerce, walter2008model} and negotiation 
\cite{rahwan2002intelligent, maes1999agents}. No relevant literature was found 
regarding simulating user behaviour in websites with agents.

% \section{Multi-agent interaction}

% \todo{focus on emergence and interaction between agents}

\section{Discrete Event Simulation (DES)} \label{ssec:des}

A discrete event simulation (DES) models a process as a series of discrete 
events, where the state of the system changes only at well defined points in 
time \cite{Siebers2010}. It was originally proposed by Kiviat in 1969 
\cite{Kiviat1969} and there is extensive research in this simulation technique. 
Banks et al. \cite{Banks2004} provides a comprehensive description and analysis 
of DES. The algorithm \ref{alg:des} is a possible implementation of a very 
simple and single-threaded DES.

\begin{algorithm}[h]
    \caption{Basic DES algorithm}
    \label{alg:des}
    %later in the document
    \begin{algorithmic}
        \State $EndCondition \gets false$
        \State $Clock \gets 0$
        \State $EventList \gets initialEvent$
        \While{$EndCondition = false$}
        \State $CurrentEvent \gets \Call{Pop}{EventList}$
        \State $Clock \gets \Call{Time}{CurrentEvent}$
        \State $\Call{Execute}{CurrentEvent}$ \Comment{might put new 
            events in $EventList$}
        \State $\Call{UpdateStatistics}$
        \EndWhile
        \State $\Call{GenerateReport}$
    \end{algorithmic}
\end{algorithm}

The major concepts in DES are~\cite{Banks2004}:

\begin{itemize}
    \item Entity, objects explicitly represented in the model (e.g a customer);
    \item Event, an occurrence that changes the state of the system (e.g a  
    customer enters the website);
    \item Event list (or future event list or pending event set), a list of 
    future events, ordered by time of occurrence;
    \item Clock, used to keep track of the current simulation time.
\end{itemize}

Event list is one of the fundamental parts of the system and it has been widely 
researched \cite{Henriksen1986, Jones1986, Tan2000, Dickman2013}.

Pidd \cite{pidd1998computer} proposes a three-phased approach that consists of: 
jump to the next chronological event, executing all the unconditional events 
(or type B) that happen that moment and then executing all the conditional 
events (or type C). This approach has advantages in terms of less usage of 
resources compared to other simplistic approaches. Also, there has 
been studies on how to scale DES to distributed and parallel (PDES)
executions~\cite{Misra1986, Fujimoto1990}.

\cite{SiebersDES2010} states that ``DES is useful for problems (...) in which 
the processes can be well defined and their emphasis is on representing 
uncertainty through stochastic distributions'', which makes DES a good 
candidate to model the problem at hand.

\section{Hybrid and novel approaches}

In recent years, there has been research which proposes a marriage between 
agent based model and simulation with discrete event simulation, however, this 
concept is not widely recognized \cite{Brailsford2014}. Brailsford states that 
the line that divides agent based models (and simulation) and DES is spurious 
and that common distinctions between the two approaches are artificial. Casas 
et al. \cite{FonsecaiCasas2011} describe a method where multi agent system 
components have been added to an existing discrete event simulation implemented 
in OMNeT++\footnote{C++ based discrete event simulation 
    toolkit}\cite{Varga2001}. Onggo \cite{Onggo2007} shows how agent based 
    models 
can be ran on top of a DES engine. Kurve et al.~\cite{Kurve2013} describes an 
agent based performance model of a PDES kernel. Regarding existing software, 
AnyLogic claims to be ``the only simulation tool that supports Discrete Event, 
Agent Based, and System Dynamics Simulation''~\cite{AnyLogic2000}. AnyLogic was 
first shown in 2000 at the Winter Simulation Conference.

\section{Summary}

This literature reviews shows that there is vast research regarding simulation, 
either agent based or DES, however not everyone is speaking the same language. 
The extensions to DES seen above are particularly interesting since they can be 
used to scale the simulation to a greater number of entities as well as 
modelling real world processes with more fidelity.

% \section{Recommender systems}

% \todo{recommender systems} % 
%http://www.ibm.com/developerworks/library/os-recommender1/index.html

% \section{A/B testing \& statistics}

% \todo{A/B}
% 
%https://yanirseroussi.com/2016/06/19/making-bayesian-ab-testing-more-accessible/
% http://conversionxl.com/statistical-significance-does-not-equal-validity/
