\chapter{System Simulation} \label{chap:simulation}

\section*{}

This chapter intends to introduce some approaches to computational simulation 
systems and engines, namely agent based and discrete event simulation. To 
finish the chapter, we show some novel approaches to simulation.

\section{Introduction}

Simulations are used to reproduce the behaviour of a system. They have been 
applied to different areas like physics, weather, biology, economics and many 
others. There are many types of simulations: stochastic or deterministic, 
steady-state or dynamic, continuous or discrete and local or distributed 
\cite{WKSimulation}. These categories are not exhaustive nor exclusive.

In this literature review, we are particularly interested in studying 
simulations which can model stochastic processes and not dynamic (dynamic 
systems are usually described by differential equations and are continuous).

\section{Agent Based Simulation (ABS)}

In agent based simulation (ABS), sometimes described as agent based computing 
\cite{wooldridge1998agent, jennings1999agent}, the individual entities in the 
model are represented discretely and maintain a set of behaviours, beliefs or 
rules that determine how their state is updated. \cite{Niazi2011} lists three 
different approaches to agent based modelling and simulation:

\begin{itemize}
    \item \emph{Agent-oriented programming} which puts emphasis on developing 
    complex individual agents rather than a large set of agents;
    \item \emph{Multi-agent oriented programming} focus on adding \emph{some} 
    intelligence to agents and observe their interactions;
    \item \emph{Agent-based or massively multi-agent modelling} where the main 
    idea is to build simple models for the agents which interact with a large 
    population of other agents to observe the global behaviour.
\end{itemize}

\cite{Siebers2010} describes ABS as ``well suited to modelling systems with 
heterogeneous, autonomous and pro-active actors, such as human-centred 
systems.'', which make them a good candidate to be used in the development of 
this dissertation. However, existing literature is quite confusing and broad, 
using different terms to refer to the same concepts, without clear 
distinctions between different agent based approaches and without consensus 
\cite{Niazi2011, Brailsford2014}.

Many platforms and frameworks were developed to support agent-based modelling 
and similation. Some notable examples include Repast~\cite{collier2003repast}, 
NetLogo~\cite{wilensky1999netlogo}, StarLogo~\cite{resnick1996starlogo} or 
MASON~\cite{panait2005cooperative}. An updated list is maintained at OpenABM 
\cite{OpenABM2016}.

Agents have been applied to e-commerce context mostly in two distinct areas: 
recommendation systems \cite{xiao2007commerce, walter2008model} and negotiation 
\cite{rahwan2002intelligent, maes1999agents}. No relevant literature was found 
regarding simulating user behaviour in websites with agents.

\section{Discrete Event Simulation (DES)}



\cite{Siebers2010}

\section{Hybrid and novel approaches}

\cite{Kurve2013}

\cite{Brailsford2014}

\cite{FonsecaiCasas2011}

\cite{Onggo2007}

\section{Summary}
