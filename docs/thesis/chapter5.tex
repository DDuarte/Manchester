\chapter{Conclusion and Future Work} \label{chap:conclusion}

\section*{}

This chapter concludes the work realized for the dissertation, it presents the 
main contributions and proposes future work on the framework.

% \section{Overview}

% The state of the art literature review focused on three main areas: 
%e-commerce, 
% simulation and probabilistic models. There was an attempt at showing the 
% classic techniques and approaches but also in showing extensions and 
% improvements to those. The validation was done by both fabricating scenarios 
% with clear expected results and with real data.

\section{Overview \& Main Contributions}

The main results obtained in the realization of this dissertation are 
summarized as follows:

\begin{itemize}
    \item Implementation of a framework capable of running a multi-agent 
    simulation applied to the problem of modelling users interacting with an 
    e-commerce website;
    \item A novel way to exploit multi-agent interaction, in this context, by 
    using two representation of agents, the navigation agents (i.e users) and 
    website agents (i.e recommendation engines), which interact with each other 
    indirectly and form a feedback loop;
    \item A tool which appeals to both the academic community and the industry. 
    The framework can be used, for example, to validate and test recommendation 
    engines and algorithms or be used by an e-commerce company to optimize 
    their own platform (e.g A/B testing);
    \item Implementation available to the community, licensed under MIT and 
    hosted on GitHub\footnote{\url{https://github.com/DDuarte/Manchester}}.
\end{itemize}

\section{Future Work}

The implementation of the framework should be seen as a foundation for further 
development. There are certain limitations and assumptions in the developed 
model that should be resolved in order for the tool be even more useful and 
usable than it currently is. In no particular order, we list some ideas for the 
future:

\begin{itemize}
    \item \textbf{Parallel simulator}. The implemented discrete event simulator 
    is single threaded and it consumes the events sequentially, which limits 
    the size of the simulation as seen in section \ref{sec:scalability}. The 
    implementation could change to a parallel simulator which hopefully 
    processes more events in the same time frame, taking advantage of 
    multi-core setups.
    
    \item \textbf{More metrics}. We have implemented only a handful of metrics 
    to be calculated after each simulation run, however, there is a myriad of 
    other metrics and statistics that we have not looked at. Further 
    development could increase the pool of available metrics.
    
    \item \textbf{Metrics extensibility}. Related to the point above, current 
    implementation \textit{hardcodes} the calculation of certain metrics in the 
    framework itself and it is not very practical to extend and add new metrics 
    to the system. The framework could be modified to ease the process of 
    adding new metrics, the visitor design pattern\cite{gamma1995design} seems 
    particularly well suited for this task.
    
    \item \textbf{Metrics for website agents}. While there is plenty of metrics 
    for the navigation agents (i.e users/consumers), metrics for website agents 
    (i.e recommendation engines) were overlooked and are not present in the 
    current implementation. It might be useful to gather metrics and statistics 
    regarding the behaviour of website agents.
    
    \item \textbf{Hypothesis testing}. Especially relevant when comparing 2 
    simulation runs, simply comparing single numeric metrics side by side might 
    not be the best approach. In the field of statistical hypothesis testing 
    (e.g A/B testing) there has been plenty of research in which standard tests 
    to use for each case. For example, to compare conversion rates Fisher's 
    exact test could be used however to compare the number of products bought a 
    $\chi^2$ test would be more appropriate \cite{wikiab}.
    
    \item \textbf{Limited number of actions}. The actions emitted by the 
    navigation agents are finite and not exhaustive. The framework does not 
    currently support extending the number of actions. % example commenting?
    
    \item \textbf{Visual aspects}. Pages are currently represented by their 
    name/URL, tags and links, leaving no space to represent visual information 
    and other meta-data. A navigation agent cannot use visual properties of the 
    pages or products (usability, aesthetics) to decide on which action to do. 
    If the intention is to model human behaviour with fidelity, this might be a 
    major hindrance. A future version of the framework should take into account 
    these aspects however it requires further research, since it is not obvious 
    how to model and represent this.
    
    % http://martinfowler.com/eaaDev/EventSourcing.html
    % http://martinfowler.com/bliki/CQRS.html
    % https://github.com/strongtyped/fun-cqrs
\end{itemize}
