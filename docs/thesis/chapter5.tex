\chapter{Conclusion and Future Work} \label{chap:conclusion}

\todo{rewrite whole section}

\section*{}

This chapter concludes the work realized for the dissertation planning. It 
presents the conclusions including a SWOT analysis.

\section{Overview}

The state of the art literature review focused on three main areas, e-commerce, 
simulation and probabilistic models. There was an attempt at showing the 
classic techniques and approaches but also in showing extensions and 
improvements to those.

In order to better assess the work required for the dissertation, a SWOT 
analysis is in order.

% \subsection{SWOT Analysis}
% 
% Regarding strengths, there is a wide and vast research in the area of 
% simulation and modelling. There is also a strong confidence that the methods, 
% approaches and techniques reviewed in the state of the art can be used to 
%model 
% user browsing behaviour successfully.
% 
% Concerning weaknesses, the proposed framework relates too many different 
%areas, 
% which may make the dissertation too broad in scope, making it hard to focus 
%on 
% something specific (``do one thing and do it well''\footnote{Unix 
%philosophy}).
% 
% In regards to opportunities, no other similar framework or tool was found, 
% which means that the work being done might be novel and may provide extra 
%value 
% to third parties (e.g the scientific community as tool to validate and test 
% others models like recommendation systems or to companies focusing on 
% e-commerce).
% 
% Regarding threats, testing and validating the framework may be problematic, 
%in 
% terms of corroborating results with real data or even finding a suitable 
% approach to validation itself.


\section{Main Contributions}

\section{Future Work}

The implementation of the framework should be seen as a foundation for further 
development. There are certain limitations and assumptions in the developed 
model that should be resolved in order for the tool be even more useful and 
usable than it currently is. In no particular order, we list some ideas for the 
future:

\begin{itemize}
    \item \textbf{Parallel simulator}. The implemented discrete event simulator 
    is single threaded and it consumes the events sequentially, which limits 
    the size of the simulation as seen in section \ref{sec:scalability}. The 
    implementation could change to a parallel simulator which hopefully 
    processes more events in the same time frame, taking advantage of 
    multi-core setups.
    
    \item \textbf{More metrics}. We have implemented only a handful of metrics 
    to be calculated after each simulation run, however, there is a myriad of 
    other metrics and statistics that we have not looked at. Further 
    development could increase the pool of available metrics.
    
    \item \textbf{Metrics extensibility}. Related to the point above, current 
    implementation \textit{hardcodes} the calculation of certain metrics in the 
    framework itself and it is not very practical to extend and add new metrics 
    to the system. The framework could be modified to ease the process of 
    adding new metrics, the visitor design pattern\cite{gamma1995design} seems 
    particularly well suited for this task.
    
    \item \textbf{Metrics for website agents}. While there is plenty of metrics 
    for the navigation agents (i.e users/consumers), metrics for website agents 
    (i.e recommendation engines) were overlooked and are not present in the 
    current implementation. It might be useful to gather metrics and statistics 
    regarding the behaviour of website agents.
    
    \item \textbf{Hypothesis testing}. Especially relevant when comparing 2 
    simulation runs, simply comparing single numeric metrics side by side might 
    not be the best approach. In the field of statistical hypothesis testing 
    (e.g A/B testing) there has been plenty of research in which standard tests 
    to use for each case. For example, to compare conversion rates Fisher's 
    exact test could be used however to compare the number of products bought a 
    $\chi^2$ test would be more appropriate \cite{wikiab}.
\end{itemize}

\todo{http://martinfowler.com/eaaDev/EventSourcing.html \& CQRS}

\todo{take into account visual aspects of the pages}

\todo{limited/hardcoded number of actions}

\todo{test statistics in sim results}
