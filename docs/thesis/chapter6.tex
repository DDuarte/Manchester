\chapter{Conclusion and Future Work} \label{chap:conclusion}

\section*{}

% Este capítulo pode ser dedicado à apresentação de detalhes de nível
% mais baixo relacionados com o enquadramento e implementação das
% soluções preconizadas no capítulo anterior.
% Note-se no entanto que detalhes desnecessários à compreensão do
% trabalho devem ser remetidos para anexos.

% Dependendo do volume, a avaliação do trabalho pode ser incluída neste
% capítulo ou pode constituir um capítulo separado.

\section{Conclusion}

\section{Future Work}

\section{SWOT Analysis}

In order to better assess the work required for the dissertation, a SWOT 
analysis is in order.

Regarding strengths, there is a wide and vast research in the area of 
simulation and modelling. There is also a strong confidence that the methods, 
approaches and techniques reviewed in the state of the art can be used to model 
user browsing behaviour successfully.

Concerning weaknesses, the proposed framework relates to many different areas, 
which may make the dissertation too broad in scope, making it hard to focus on 
something specific (``do one thing and do it well''\footnote{Unix philosophy}).

In regards to opportunities, no other similar framework or tool was found, 
which means that the work being done might be novel and may provide extra value 
to third parties (e.g the scientific community as tool to validate and test 
others models like recommendation systems or to companies focusing on 
e-commerce).

Regarding threats, testing and validating the framework may be problematic, in 
terms of corroborating results with real data or even finding a suitable 
approach to validation itself.
